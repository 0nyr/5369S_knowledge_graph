% !TeX spellcheck = en_US
% !TeX encoding = UTF-8
% !TEX program = xelatex

% VSCODE word wrap: ALT + Z
% COMPILE WITH:
% `latexmk`
% latexmk -lualatex -pdf main.tex
% You need lualatex and biber (in all TeXLive distributions)

\documentclass[12pt]{report} % text width
\usepackage[utf8]{inputenc} % encode text to utf8

% paragraph formatting: https://www.overleaf.com/learn/latex/Paragraph_formatting
\setlength{\parindent}{1em}
\setlength{\parskip}{1em}

% better language support
\usepackage{polyglossia}
\setdefaultlanguage{english}
\setotherlanguage{german}

% use lualatex or xelatex
\usepackage{fontspec} % font selection
\usepackage[a4paper]{geometry}
\usepackage{graphicx}% for graphics
\usepackage[onehalfspacing]{setspace}
\usepackage{tocbasic}
\usepackage{booktabs}
\usepackage{multicol}
\usepackage{multirow}
\usepackage[]{scrlayer-scrpage}

% quotes and bibliography: https://www.overleaf.com/learn/latex/Typesetting_quotations
\usepackage[
    left = \flqq{},% 
    right = \frqq{},% 
    leftsub = \flq{},% 
    rightsub = \frq{} %
]{dirtytalk}
\usepackage{csquotes}
\usepackage[
    backend=biber,
    style=numeric,
    sorting=none
]{biblatex}
% add commands for automatic cite/uncite distinction
\DeclareBibliographyCategory{cited}
\AtEveryCitekey{\addtocategory{cited}{\thefield{entrykey}}}
\addbibresource{biblio.bib} % bibliography
\nocite{*} % all references

\newcommand{\ts}{\textsuperscript} % superscript for 2nd or XIXème

\pagenumbering{roman} % set page numbering of front matter sections

% use acronyms and glossaries
% toc: add glossary to table of contents
\usepackage{hyperref}
\usepackage[acronym, toc]{glossaries} 
\makeglossaries
\newglossaryentry{nodes}
{
    name=nodes,
    description={A node is an entity in a graph, it can be a person, a place, a thing, or any other entity.}
}

\newacronym{kg}{KG}{Knownledge Graph}
\newacronym{foss}{FOSS}{Free and Open Source Software}
%\input{glossaries.tex} % acronyms definitions, failed to make in work on a separate file!!!

% custom commands
% escape char in latex: https://tex.stackexchange.com/questions/34580/escape-character-in-latex
% horizontal spacing: https://tex.stackexchange.com/questions/74353/what-commands-are-there-for-horizontal-spacing/74354
\newcommand{\p}{\texttt{+}} % small unary plus
\newcommand{\doublep}{\texttt{++}} % double small unary plus
\newcommand{\m}{\texttt{-} \space} % small unary minus
\newcommand{\doublem}{\texttt{-}\texttt{-} \space} % double small unary minus

% code 
\usepackage{listings}
\usepackage{hyphenat} % fix "overfull hbox" with slicing words using hyphenation
\hyphenation{hy-phen-a-tion} % indicate all 3 permissible hyphenation points

% where to put all images and figures
\graphicspath{{img/}}


% document info
\title{What is a Knownledge Graph?}
\author{Rascoussier, Florian Guillaume Pierre}
\date{April-Mai 2023}

% document content
\begin{document}

% !TeX spellcheck = en_US
% !TeX encoding = UTF-8
\begin{titlepage}
    \centering
    \begin{onehalfspace}
    	
        	\includegraphics[width=7cm]{uni-logo.png}\\
        	\vspace{1.0cm}
        	\large {\bfseries 5369S Seminar: Knowledge Graphs }\\

        	\vspace{2.5cm}

            \begin{doublespace}
            	{\textsf{\Huge{\thetitle}}}
            \end{doublespace}

        	\vspace{2cm}

            \Large{Report}\\

        	\vspace{1cm}

        	{\bfseries \large{\theauthor}}

        	\vfill

        	{\large
                \textsc{1.~Pr\"ufer} \\
                Prof. Dr. Alsayed Algergawy
        	}

        	\vspace{1.5cm}

        	\parbox{\linewidth}{\hrule\strut}

            \vfill

	    \thedate
    \end{onehalfspace}
\end{titlepage}

\newpage

%%%%%%%%%%%%%%%%%%%%%%%%%%%%%%%%%%%%%%%%%%%%%%%%%%%%%%%%%%%%%%%%%%%%%%%%%%%%%%%%%%%%%%%%%
\frontmatter
% CHOOSE ACCORDINGLY
%\include{includes/BA-titlepage}
\include{includes/MA-titlepage}

\tableofcontents
\newpage

% -- ABSTRACT
\section*{Abstract}
Knowledge graphs (\gls{KG}s) represent a powerful approach to organizing and structuring real-world information by modeling entities, their properties, and the relationships between them. As an enabling technology, KGs have gained significant traction in various domains such as natural language processing, information retrieval, recommendation systems, and semantic search. This paper provides a comprehensive introduction to knowledge graphs, outlining their use cases, the current state of research, and industry adoption.

By facilitating advanced querying, reasoning, and knowledge discovery, KGs have become instrumental in numerous applications. The integration of KGs with machine learning techniques, such as graph neural networks and entity embeddings, has further bolstered their capabilities in prediction and pattern recognition. Research efforts are concentrated on KG construction, embedding methods, reasoning techniques, and evaluation metrics, while addressing issues like scalability, incompleteness, and dynamic evolution.

In the industry, major technology companies, including Google, Microsoft, and Facebook, have embraced KGs to enhance search engines, virtual assistants, and social media platforms. A rising number of startups and specialized firms are also employing KGs for diverse applications, ranging from drug discovery to fraud detection and smart manufacturing. Despite the considerable progress, challenges persist in areas such as data validation, real-time updates, privacy preservation, and usability. The current report discusses what are Knowledge Graphs, and introduces related concepts like \gls{KG} construction, embedding methods, reasoning techniques, and evaluation metrics, while addressing issues like scalability, incompleteness, and dynamic evolution. It also outlines the current state of research, industry adoption, and future directions to advance the adoption and impact of knowledge graphs.

% -- Acknowledgements
\section*{Acknowledgements}
I would like to express my sincere gratitude to all the participants of the seminar on Knowledge Graph for their valuable insights, discussions, and contributions. I am especially grateful to the esteemed lecturers, Prof. Dr. Alsayed Algergawy, Asha Mannarapotta Venugopal, and Vishvapalsinhji Ramsinh Parmar, whose expertise and guidance have been instrumental in shaping my understanding of this crucial topic. 

A special acknowledgment goes to my monitor, Prof. Dr. Alsayed Algergawy, for his unwavering support, encouragement, and inspiration throughout the course of this work. His knowledge and experience in the field have been invaluable in this seminar. I am deeply appreciative of his mentorship and the opportunity to learn from a distinguished expert in the realm of knowledge graphs.

\newpage

% -- List of figures
\thispagestyle{empty}
\cleardoublepage
\listoffigures
\newpage

% -- List of tables
\thispagestyle{empty}
\cleardoublepage
\listoftables
\newpage

%%%%%%%%%%%%%%%%%%%%%%%%%%%%%%%%%%%%%%%%%%%%%%%%%%%%%%%%%%%%%%%%%%%%%%%%%%%%%%%%%%%%%%%%%
\mainmatter

\section{Introduction}


\section{What is a Knowledge Graph?}

\subsection{Definition}

\subsection{History}

\subsection{Use cases}

\subsection{Knowledge Graphs vs. Semantic Web}


\section{Advances topics in Knowledge Graphs}

\subsection{Construction}

\subsection{Search and Reasoning}

\subsection{Schema and Ontology}

\subsection{Machine Learning}


\section{Research and Industry}


%%%%%%%%%%%%%%%%%%%%%%%%%%%%%%%%%%%%%%%%%%%%%%%%%%%%%%%%%%%%%%%%%%%%%%%%%%%%%%%%%%%%%%%%%
\backmatter

% glossary and acronyms
\newpage
\printglossary[type=\acronymtype, title=Acronymes]

\printglossary[title=Glossaire]

% biblio
\newpage
\printbibliography[
    heading=bibintoc,
    category=cited,
    title={References}
]

% uncited references (bibliography)
% https://tex.stackexchange.com/questions/6967/how-to-split-bibliography-into-works-cited-and-works-not-cited
\printbibliography[
    notcategory=cited,
    heading=bibintoc,
    title={Additional bibliography},
]


\restoregeometry
\end{document}
